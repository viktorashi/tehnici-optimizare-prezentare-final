\documentclass{purdue-slide}

% For filler text:
\usepackage[base]{babel}
\usepackage{lipsum}
\usepackage{lmodern}
\usepackage{pgfplots}
\usepackage{caption}
\usepackage{tikz}
\usetikzlibrary{positioning}

% \renewcommand{\sfdefault}{cmss} % Use Computer Modern Sans
\renewcommand{\sfdefault}{phv} % Use Helvetica

\subtitle{Solving MILP problems using the GLPK solver.}
\author{Stan Ioan-Victor, Ioan-Gabriel Spatariu, Ana Sabina Tatar, Mihnea-Gabriel Vasile, Tepfenhart Bertold }

\begin{document}

\begin{titleframe}{PuLP with Python}
	\maketitle
\end{titleframe}

\begin{frame}{Contents}
	% \textbf{Introduction to Linear Programming} \\
	% \textbf{Mathematical Theory \& Theorems} \\
	% \textbf{Introduction to PuLP} \\
	% \textbf{Case Studies} \\
	% \textbf{Resources \& References} \\
	\tableofcontents
\end{frame}

\section{Theoretical Foundation}

\begin{frame}{Fundamental Theorem of LP}
	\textbf{Theorem:} If a linear programming (LP) problem is feasible and bounded, then it has an optimal solution.
\end{frame}

\begin{frame}{Proof of the Theorem}
	Without loss of generality, we may assume that the LP problem is of the form:
	\[
		\begin{aligned}
			& \min \mathbf{c}^\top \mathbf{x} \\
			(P) \quad & \text{s.t. } A\mathbf{x} \geq \mathbf{b}
		\end{aligned}
	\]
	where $m$ and $n$ are positive integers,  $A \in \mathbb{R}^{m \times n}, \ \mathbf{b} \in \mathbb{R}^m, \mathbf{c} \in \mathbb{R}^n,$ $ \text{ and }
	\mathbf{x} =
	\begin{bmatrix}
		x_1 \\
		\vdots \\
		x_n
	\end{bmatrix} \text{ is a tuple of variables. }$

\end{frame}

\begin{frame}{Transforming to Standard Form}
	\quad Any LP problem can be converted to the standard form $(P)$ having the same feasible region and optimal solution set.

	For this, note that:

	\begin{itemize}
		\item Constraints of the form $\mathbf{a}^\top \mathbf{x} \leq \beta$ become $-\mathbf{a}^\top \mathbf{x} \geq -\beta$
		\item Constraints of the form  $\mathbf{a}^\top \mathbf{x} = \beta$ become:
			\[
    \mathbf{a}^\top \mathbf{x} \geq \beta \quad \text{and} \quad -\mathbf{a}^\top \mathbf{x} \leq -\beta
			\]
		\item Maximization problems become minimizations of $-\mathbf{c}^\top \mathbf{x}$
	\end{itemize}
\end{frame}

\begin{frame}{Auxiliary System}
	Suppose $(P)$ is feasible. Define an auxiliary system:
	\[
		(S) \quad
		\begin{cases}
			z - \mathbf{c}^\top \mathbf{x} \geq 0 \\
			-z + \mathbf{c}^\top \mathbf{x} \geq 0 \\
			A\mathbf{x} \geq \mathbf{b}
		\end{cases}
	\]

	\vspace{0.5em}
	Solving $(P)$ is equivalent to minimizing $z$ in $(S)$.

	\vspace{0.5em}
	We eliminate variables $x_1, \dots, x_n$, reducing $(S)$ to a system $(S')$ involving only $z$.
\end{frame}

\begin{frame}{Boundedness and Optimal Solution}
	If $(P)$ is bounded, then $(S')$ must include the constraints:
	\[
		z \geq \beta_i \text{, where } \beta_i \in \mathbb{R}, \forall  i = 1, \dots, p, \text{ with } p \in \mathbf{N^*}.
	\]
	Define $\gamma = \max\{\beta_1, \dots, \beta_p\}$. Then for any solution $x, z$ to $(S)$, $z$ is at least $\gamma$. But we can set $z = \gamma$ and extend it to a solution for $(S)$.

	Hence, we obtain an optimal solution for $(P)$ and $\gamma$ is the optimal value. This completes the proof of the theorem.
\end{frame}

\begin{frame}{ Multipliers and the Inequality}
	$\mathbf{Remark:}$ We can construct multipliers to derive the inequality:
	\[
		\mathbf{c}^\top \mathbf{x} \geq \gamma
	\]
	from the system $A\mathbf{x} \geq \mathbf{b}$.

	There exist real values $\alpha, \beta, y_1^*, \ldots, y_m^* \geq 0$ such that:
	\[
		\left[ \alpha \ \beta \ y_1^* \cdots y_m^* \right]
		\begin{bmatrix}
			1 & -\mathbf{c}^\top \\
			-1 & \mathbf{c}^\top \\
			0 & A
		\end{bmatrix}
		\begin{bmatrix}
			z \\ \mathbf{x}
		\end{bmatrix}
		\geq
		\left[ \alpha \ \beta \ y_1^* \cdots y_m^* \right]
		\begin{bmatrix}
			0 \\ 0 \\ \mathbf{b}
		\end{bmatrix}
	\]
\end{frame}

\section{Hand-written examples}

\end{document}